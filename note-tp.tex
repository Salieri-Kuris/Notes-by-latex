\documentclass[12pt, a4paper, oneside]{ctexbook}
\usepackage{amsmath, amsthm, amssymb, bm, graphicx, hyperref, mathrsfs, float}


\title{{\Huge{\textbf{凝聚态物理中的拓扑物态}}}\\}
\author{江文涛 191840114}
\date{\today}
\linespread{1.2}

\setcounter{tocdepth}{1}%目录显示到第一级,section
\setcounter{secnumdepth}{2}%编号到第二级,subsection

\newcommand{\fp}[2]{\frac{\partial#1}{\partial#2}}
\begin{document}
	\maketitle
	
	\pagenumbering{roman}
	\setcounter{page}{1}
	\newpage
	\pagenumbering{Roman}
	\setcounter{page}{1}
	\tableofcontents
	\newpage
	\setcounter{page}{1}
	\pagenumbering{arabic}

    \chapter{Berry相位}
    \section{离散情况}
    干涉条纹的产生来自$ P=|\Psi_1+\Psi_2|^2=\Psi_1^2+\Psi_2^2+\Psi_1^*\Psi_2+\Psi_1\Psi_2^* $。\\

    相对相位是有物理意义的。$ \gamma_{12}=-arg<\Psi_1|\Psi_2> $ 
	\subsection*{离散Berry相位的定义}
	对于N个Hilbert空间中的态,将它们排序为一个圈,定义下面的一个量:
	\begin{equation}
		\gamma_L=-\arg e^{-i\left(\gamma_{12}+\gamma_{23}+\ldots+\gamma_{N1}\right)}=-\arg \left(\left\langle\Psi_1 \mid \Psi_2\right\rangle\left\langle\Psi_2 \mid \Psi_3\right\rangle \ldots\left\langle\Psi_N \mid \Psi_1\right\rangle\right)
	\end{equation}
	它可以改写为如下形式:
	\begin{equation}
		\gamma_L=-\arg \operatorname{Tr}\left(\left|\Psi_1\right\rangle\left\langle\Psi_1|| \Psi_2\right\rangle\left\langle\Psi_2|\ldots| \Psi_N\right\rangle\left\langle\Psi_N\right|\right)
	\end{equation}
	作为一组规范不变投影算符的乘积,它显然也是规范不变的。
    \subsubsection*{Berry通量}
	考虑一个二维格点,将每个点都对应一个态,定义如下一个Berry相:
	\begin{equation}
		\begin{aligned}
			\gamma_L=-\arg \exp \left[-i\left(\sum_{n=1}^{N-1} \gamma_{(n, 1),(n+1,1)}\right.\right. & +\sum_{m=1}^{M-1} \gamma_{(N, m),(N, m+1)} \\
			& \left.\left.+\sum_{n=1}^{N-1} \gamma_{(n+1, M),(n, M)}+\sum_{m=1}^{M-1} \gamma_{(1, m+1),(1, m)}\right)\right]
			\end{aligned}
	\end{equation}
	直接计算涉及很多依赖规范的量,下面引入一种方法使得计算可以拆解为规范不变量的和。对于每个“小圈”,定义它边界对应的Berry相位:
	\begin{equation}
		\begin{aligned}
			F_{n m}&=-\arg \exp \left[-i\left(\gamma_{(n, m),(n+1, m)}+\right.\right.  \gamma_{(n+1, m),(n+1, m+1)} \\
			& \left.\left.+\gamma_{(n+1, m+1),(n, m+1)}+\gamma_{(n, m+1),(n, m)}\right)\right]\\
			& =\gamma_{(n, m),(n+1, m)}+\gamma_{(n+1, m),(n+1, m+1)}+\gamma_{(n+1, m+1),(n, m+1)}+\gamma_{(n, m+1),(n, m)}+2\pi n_{nm}
			\end{aligned}
	\end{equation}
	称为Berry通量。将所有的Berry通量相乘,每个内部的边都被计算了方向相反的两次,因此互为复共轭,最终只剩下边界的相,所以这个量和之前定义的Berry相相等。
	\begin{equation}
		\exp \left[-i \sum_{n=1}^{N-1} \sum_{m=1}^{M-1} F_{n m}\right]=e^{-i \gamma_L}
		\label{eq}
	\end{equation}
	这让人联想到Stokes定理,但在这里Berry通量的求和与Berry相本身并不相等,而是可以相差$ 2\pi n $。
	\subsection*{Chern数}
	对于上边定义的格点参数空间取周期边界条件,\eqref{eq}变为$ \prod_{m=1}^M \prod_{n=1}^N e^{-i F_{n m}}=1 $ ,Chern数的定义为:
	\begin{equation}
		Q=\frac{1}{2 \pi} \sum_{n m} F_{n m}
	\end{equation}
	它显然是一个整数。
	\section{连续情况}
	现在用D维参数空间中的矢量标记Hilbert空间中的态矢:$ |\Psi(\mathbf{R})\rangle $。在参数空间中取一定向曲线:$ \mathscr{C}:[0,1) \rightarrow \mathscr{P}, \quad t \mapsto \mathbf{R}(t) $,
	曲线上相邻的两点映射到的态矢之间的相对相位为:\\
	\begin{equation}
		e^{-i \Delta \gamma}=\frac{\langle\Psi(\mathbf{R}) \mid \Psi(\mathbf{R}+d \mathbf{R})\rangle}{|\langle\Psi(\mathbf{R}) \mid \Psi(\mathbf{R}+d \mathbf{R})\rangle|} ; \quad \Delta \gamma=i\left\langle\Psi(\mathbf{R})\left|\nabla_{\mathbf{R}}\right| \Psi(\mathbf{R})\right\rangle \cdot d \mathbf{R},
	\end{equation}

	上式右边中乘上$ d\vec{R} $的量为Berry联络:
	\begin{equation}
		\mathbf{A}(\mathbf{R})=i\left\langle\Psi(\mathbf{R}) \mid \nabla_{\mathbf{R}} \Psi(\mathbf{R})\right\rangle=-\operatorname{Im}\left\langle\Psi(\mathbf{R}) \mid \nabla_{\mathbf{R}} \Psi(\mathbf{R})\right\rangle
	\end{equation} 
	其中有定义:$ \left\langle\Phi \mid \nabla_{\mathbf{R}} \Psi(\mathbf{R})\right\rangle=\nabla_{\mathbf{R}}\langle\Phi \mid \Psi(\mathbf{R})\rangle $。\footnote{此内积纯虚,$ \nabla_{\mathbf{R}}\langle\Psi(\mathbf{R}) \mid \Psi(\mathbf{R})\rangle=\left\langle\Psi(\mathbf{R})|\nabla_{\mathbf{R}} \Psi(\mathbf{R})\right\rangle+(\left\langle\Psi(\mathbf{R})|\nabla_{\mathbf{R}} \Psi(\mathbf{R})\right\rangle)^*=0 $ }
	波函数的Local规范变换使得Berry联络如下变换:
	\begin{equation}
		|\Psi(\mathbf{R})\rangle \rightarrow e^{i \alpha(\mathbf{R})}|\Psi(\mathbf{R})\rangle: \quad \mathbf{A}(\mathbf{R}) \rightarrow \mathbf{A}(\mathbf{R})-\nabla_{\mathbf{R}} \alpha(\mathbf{R})
	\end{equation} 
	对于参数空间的一条闭曲线,沿着这条曲线的Berry相位为:
	\begin{equation}
		\gamma(\mathscr{C})=-\operatorname{argexp}\left[-i \oint_{\mathscr{C}} \mathbf{A} \cdot d \mathbf{R}\right]
	\end{equation}
	仿照Stokes定理,定义Berry曲率:
	\begin{equation}
		B=\left(\partial_x A_y-\partial_y A_x\right)
	\end{equation}
	此时线积分可改写为面积分:
	\begin{equation}
		\oint_{\partial \mathscr{F}} \mathbf{A} \cdot d \mathbf{R}=\int_{\mathscr{F}}\left(\partial_x A_y-\partial_y A_x\right) d x d y=\int_{\mathscr{F}} B d x d y
	\end{equation}
	如果规范的选取是连续的,对于闭合曲面上述积分为零,但若规范存在不连续的奇点,则会出现非零的Chern数。
	考虑这样一个二能级哈密顿量:$ \hat{H}(\mathbf{d})=d_x \hat{\sigma}_x+d_y \hat{\sigma}_y+d_z \hat{\sigma}_z=\mathbf{d} \cdot \hat{\sigma} $,$ \vec{d} $是挖去球心的球上一点。
	很容易看出,上述哈密顿量的平方正比于单位矩阵,其本征值也很容易得到。利用球坐标将$ \vec{d} $参数化,很容易得到本征矢:
	\begin{equation}
		\left|+_{\mathbf{d}}\right\rangle=e^{i \alpha(\theta, \varphi)}\left(\begin{array}{c}
			e^{-i \varphi / 2} \cos \theta / 2 \\
			e^{i \varphi / 2} \sin \theta / 2
			\end{array}\right)
	\end{equation}   
	对应负本征值的本征矢相应地为:$ \left|-_{\mathbf{d}}\right\rangle=e^{i \beta(\mathbf{d})}\left|+_{-\mathbf{d}}\right\rangle $\\
	选择$ \alpha,\beta $就是选择不同的规范,然而,无论如何选取规范,都不能在全局上性质良好。  
	\bibliographystyle{IEEEtran}
	\bibliography{Ref}
\end{document}