\documentclass[12pt, a4paper, oneside]{ctexbook}
\usepackage{amsmath, amsthm, amssymb, bm, graphicx, hyperref, mathrsfs, float}


\title{{\Huge{\textbf{凝聚态物理中的拓扑物态}}}\\}
\author{江文涛 191840114}
\date{\today}
\linespread{1.2}

\setcounter{tocdepth}{1}%目录显示到第一级,section
\setcounter{secnumdepth}{2}%编号到第二级,subsection

\newcommand{\fp}[2]{\frac{\partial#1}{\partial#2}}
\begin{document}
	\maketitle
	
	\pagenumbering{roman}
	\setcounter{page}{1}
	\newpage
	\pagenumbering{Roman}
	\setcounter{page}{1}
	\tableofcontents
	\newpage
	\setcounter{page}{1}
	\pagenumbering{arabic}

    \chapter{Berry相位}
    \section{离散情况}
    干涉条纹的产生来自$ P=|\Psi_1+\Psi_2|^2=\Psi_1^2+\Psi_2^2+\Psi_1^*\Psi_2+\Psi_1\Psi_2^* $。\\

    相对相位是有物理意义的。$ \gamma_{12}=-arg<\Psi_1|\Psi_2> $ 
	\subsection*{离散Berry相位的定义}
	对于N个Hilbert空间中的态,将它们排序为一个圈,定义下面的一个量:
	\begin{equation}
		\gamma_L=-\arg e^{-i\left(\gamma_{12}+\gamma_{23}+\ldots+\gamma_{N1}\right)}=-\arg \left(\left\langle\Psi_1 \mid \Psi_2\right\rangle\left\langle\Psi_2 \mid \Psi_3\right\rangle \ldots\left\langle\Psi_N \mid \Psi_1\right\rangle\right)
	\end{equation}
	它可以改写为如下形式:
	\begin{equation}
		\gamma_L=-\arg \operatorname{Tr}\left(\left|\Psi_1\right\rangle\left\langle\Psi_1|| \Psi_2\right\rangle\left\langle\Psi_2|\ldots| \Psi_N\right\rangle\left\langle\Psi_N\right|\right)
	\end{equation}
	作为一组规范不变投影算符的乘积,它显然也是规范不变的。
    \subsubsection*{Berry通量}
	考虑一个二维格点,将每个点都对应一个态,定义如下一个Berry相:
	\begin{equation}
		\begin{aligned}
			\gamma_L=-\arg \exp \left[-i\left(\sum_{n=1}^{N-1} \gamma_{(n, 1),(n+1,1)}\right.\right. & +\sum_{m=1}^{M-1} \gamma_{(N, m),(N, m+1)} \\
			& \left.\left.+\sum_{n=1}^{N-1} \gamma_{(n+1, M),(n, M)}+\sum_{m=1}^{M-1} \gamma_{(1, m+1),(1, m)}\right)\right]
			\end{aligned}
	\end{equation}
	直接计算涉及很多依赖规范的量,下面引入一种方法使得计算可以拆解为规范不变量的和。对于每个“小圈”,定义它边界对应的Berry相位:
	\begin{equation}
		\begin{aligned}
			F_{n m}&=-\arg \exp \left[-i\left(\gamma_{(n, m),(n+1, m)}+\right.\right.  \gamma_{(n+1, m),(n+1, m+1)} \\
			& \left.\left.+\gamma_{(n+1, m+1),(n, m+1)}+\gamma_{(n, m+1),(n, m)}\right)\right]\\
			& =\gamma_{(n, m),(n+1, m)}+\gamma_{(n+1, m),(n+1, m+1)}+\gamma_{(n+1, m+1),(n, m+1)}+\gamma_{(n, m+1),(n, m)}+2\pi n_{nm}
			\end{aligned}
	\end{equation}
	称为Berry通量。将所有的Berry通量相乘,每个内部的边都被计算了方向相反的两次,因此互为复共轭,最终只剩下边界的相,所以这个量和之前定义的Berry相相等。
	\begin{equation}
		\exp \left[-i \sum_{n=1}^{N-1} \sum_{m=1}^{M-1} F_{n m}\right]=e^{-i \gamma_L}
		\label{eq}
	\end{equation}
	这让人联想到Stokes定理,但在这里Berry通量的求和与Berry相本身并不相等,而是可以相差$ 2\pi n $。
	\subsection*{Chern数}
	对于上边定义的格点参数空间取周期边界条件,\eqref{eq}变为$ \prod_{m=1}^M \prod_{n=1}^N e^{-i F_{n m}}=1 $ ,Chern数的定义为:
	\begin{equation}
		Q=\frac{1}{2 \pi} \sum_{n m} F_{n m}
	\end{equation}
	它显然是一个整数。
	\section{连续情况}
	现在用D维参数空间中的矢量标记Hilbert空间中的态矢:$ |\Psi(\mathbf{R})\rangle $。在参数空间中取一定向曲线:$ \mathscr{C}:[0,1) \rightarrow \mathscr{P}, \quad t \mapsto \mathbf{R}(t) $,
	曲线上相邻的两点映射到的态矢之间的相对相位为:\\
	\begin{equation}
		e^{-i \Delta \gamma}=\frac{\langle\Psi(\mathbf{R}) \mid \Psi(\mathbf{R}+d \mathbf{R})\rangle}{|\langle\Psi(\mathbf{R}) \mid \Psi(\mathbf{R}+d \mathbf{R})\rangle|} ; \quad \Delta \gamma=i\left\langle\Psi(\mathbf{R})\left|\nabla_{\mathbf{R}}\right| \Psi(\mathbf{R})\right\rangle \cdot d \mathbf{R},
	\end{equation}

	上式右边中乘上$ d\vec{R} $的量为Berry联络:
	\begin{equation}
		\mathbf{A}(\mathbf{R})=i\left\langle\Psi(\mathbf{R}) \mid \nabla_{\mathbf{R}} \Psi(\mathbf{R})\right\rangle=-\operatorname{Im}\left\langle\Psi(\mathbf{R}) \mid \nabla_{\mathbf{R}} \Psi(\mathbf{R})\right\rangle
	\end{equation} 
	其中有定义:$ \left\langle\Phi \mid \nabla_{\mathbf{R}} \Psi(\mathbf{R})\right\rangle=\nabla_{\mathbf{R}}\langle\Phi \mid \Psi(\mathbf{R})\rangle $。\footnote{此内积纯虚,$ \nabla_{\mathbf{R}}\langle\Psi(\mathbf{R}) \mid \Psi(\mathbf{R})\rangle=\left\langle\Psi(\mathbf{R})|\nabla_{\mathbf{R}} \Psi(\mathbf{R})\right\rangle+(\left\langle\Psi(\mathbf{R})|\nabla_{\mathbf{R}} \Psi(\mathbf{R})\right\rangle)^*=0 $ }
	波函数的Local规范变换使得Berry联络如下变换:
	\begin{equation}
		|\Psi(\mathbf{R})\rangle \rightarrow e^{i \alpha(\mathbf{R})}|\Psi(\mathbf{R})\rangle: \quad \mathbf{A}(\mathbf{R}) \rightarrow \mathbf{A}(\mathbf{R})-\nabla_{\mathbf{R}} \alpha(\mathbf{R})
	\end{equation} 
	对于参数空间的一条闭曲线,沿着这条曲线的Berry相位为:
	\begin{equation}
		\gamma(\mathscr{C})=-\operatorname{argexp}\left[-i \oint_{\mathscr{C}} \mathbf{A} \cdot d \mathbf{R}\right]
	\end{equation}
	仿照Stokes定理,定义Berry曲率:
	\begin{equation}
		B=\left(\partial_x A_y-\partial_y A_x\right)
	\end{equation}
	此时线积分可改写为面积分:
	\begin{equation}
		\oint_{\partial \mathscr{F}} \mathbf{A} \cdot d \mathbf{R}=\int_{\mathscr{F}}\left(\partial_x A_y-\partial_y A_x\right) d x d y=\int_{\mathscr{F}} B d x d y
	\end{equation}
	如果规范的选取是连续的,对于闭合曲面上述积分为零,但若规范存在不连续的奇点,则会出现非零的Chern数。
	\section{二能级体系为例}
	考虑这样一个二能级哈密顿量:$ \hat{H}(\mathbf{d})=d_x \hat{\sigma}_x+d_y \hat{\sigma}_y+d_z \hat{\sigma}_z=\mathbf{d} \cdot \hat{\sigma} $,$ \vec{d} $是挖去球心的球上一点。
	很容易看出,上述哈密顿量的平方正比于单位矩阵,其本征值也很容易得到。利用球坐标将$ \vec{d} $参数化,很容易得到本征矢:
	\begin{equation}
		\left|+_{\mathbf{d}}\right\rangle=e^{i \alpha(\theta, \varphi)}\left(\begin{array}{c}
			e^{-i \varphi / 2} \cos \theta / 2 \\
			e^{i \varphi / 2} \sin \theta / 2
			\end{array}\right)
	\end{equation}   
	对应负本征值的本征矢相应地为:$ \left|-_{\mathbf{d}}\right\rangle=e^{i \beta(\mathbf{d})}\left|+_{-\mathbf{d}}\right\rangle $\\
	选择$ \alpha,\beta $就是选择不同的规范,然而,无论如何选取规范,都不能在全局上性质良好。  
	Dirac模型:$ \hat{H}(\mathbf{d})=d_x \hat{\sigma}_x+d_y \hat{\sigma}_y+M \hat{\sigma}_z $
	,$ E_\pm=\pm\sqrt{k^2+M^2} $  
	\subsubsection{能带反转}
	对于两能级体系,考虑$ \Gamma $点 $ m>0,m<0 $附近的能带变化,发现当能隙关闭时,高低能量态对应的轨道发生变换。
	\subsubsection{算个陈数}
	在前面利用Stokes定理,似乎陈数都是零,但这建立在被积函数光滑的条件下,当无法选取一个全局光滑的规范时,
	上述论证不成立,此时就隐含着陈数不为零的可能。对于二能级系统,一个常用的规范变换是$ \alpha(R)=\phi $,
	将参数空间分为两个规范的部分,最终的Berry曲率的积分转化为线积分时,由于规范的不光滑性,最终剩下一个$ \nabla\phi $的线积分,
	给出1的陈数。
	\subsubsection{从含时薛定谔方程导出Berry相}
	考虑这样一个含参数的哈密顿量$ H\left(x^a ; \lambda^i\right) $,$ x^a $是动力学自由度,哈密顿量通过依赖于参数$ \lambda $而依赖于时间。考虑这个哈密顿量的基态,假设为非简并的$ |\psi> $,它也依赖于参数$ \lambda $,$ |\psi(\lambda(t))> $。由于绝热定理,当参数改变的足够缓慢且哈密顿的能级之间未发生交错时,$ H(0) $对应第n个本征值的态将也演化到$ H(\lambda(t)) $的第n个本征值对应的本征态。\\

	可以看到,当改变参数使得非简并的态之间发生了能级简并时,给出Berry曲率的奇异性。上式只包含了两两之间的能量差,所以一个两能级物理体系就能体现很多问题的核心。这里通过考虑一两能级体系来看看在能量简并点到底发生了什么:对于$ E(\mathbf{R})_+\geq E(\mathbf{R})_- $,在能级简并点$ E(\mathbf{R^*})_+=E(\mathbf{R^*})_- $,将哈密顿量在简并点附近展开为:$H(\mathbf{R}) \approx H\left(\mathbf{R}^*\right)+\left(\mathbf{R}-\mathbf{R}^*\right) \cdot \boldsymbol{\nabla} H\left(\mathbf{R}^*\right)$,这给出Berry曲率:
\begin{equation}
	\mathbf{V}_{+}((\mathbf{R}))=\operatorname{lm} \frac{\left\langle+(\mathbf{R})\left|\left(\nabla H\left(\mathbf{R}^*\right)\right)\right|-(\mathbf{R})\right\rangle \times\left\langle-(\mathbf{R})\left|\left(\nabla H\left(\mathbf{R}^*\right)\right)\right|+(\mathbf{R})\right\rangle}{\left(E_{+}(\mathbf{R})-E_{-}(\mathbf{R})\right)^2} .
\end{equation} 
由于叉乘的反对易性,很容易看出两个能级的Berry曲率刚好为相反数。具体而言,考虑这样一个二能级哈密顿量:$ \hat{H}(\mathbf{d})=d_x \hat{\sigma}_x+d_y \hat{\sigma}_y+d_z \hat{\sigma}_z=\mathbf{d} \cdot \hat{\sigma} $,$\mathbf{d}$。\\

很容易看出,上述哈密顿量的平方正比于单位矩阵,其本征值也很容易得到。利用球坐标将$ \vec{d} $参数化,很容易得到本征矢:
\begin{equation}
	\left|+_{\mathbf{d}}\right\rangle=e^{i \alpha(\theta, \varphi)}\left(\begin{array}{c}
		e^{-i \varphi / 2} \cos \theta / 2 \\
		e^{i \varphi / 2} \sin \theta / 2
		\end{array}\right)
\end{equation}   
对应负本征值的本征矢相应地为:$ \left|-_{\mathbf{d}}\right\rangle=e^{i \beta(\mathbf{d})}\left|+_{-\mathbf{d}}\right\rangle $\\
不难算出,Berry曲率在此问题中为:$ \mathbf{V}_{+}(\mathbf{R})=\frac{\mathbf{R}}{2 R^3} $,具有位于原点处磁单极子的形式,而哈密顿量在参数空间的原点处简并,因此每有一个参数空间的简并点都给出该处的一个磁单极子形式的Berry曲率。 
\chapter{SSH模型}
SSH模型描述了一维双原子链中电子的跃迁,忽略电子间的相互作用与自旋,假设平移不变性,单粒子哈密顿量为:
\begin{equation}
	\hat{H}=v \sum_{m=1}^N(|m, B\rangle\langle m, A|+\text { h.c. })+w \sum_{m=1}^{N-1}(|m+1, A\rangle\langle m, B|+\text { h.c. })
\end{equation}
为方便起见,总是选取$ v,w $为实数,复数的$ v,w $总是可以对态做规范变换回到实数。\\

每个元胞中含有两个子格点,因为忽略了其他的内禀自由度,如自旋、轨道,此Hilbert空间可以等价于N个二能级系统,因此哈密顿量可以改写为:
\begin{equation}
	\hat{H}=v \sum_{m=1}^N|m\rangle\langle m| \otimes \hat{\sigma}_x+w \sum_{m=1}^{N-1}\left(|m+1\rangle\langle m| \otimes \frac{\hat{\sigma}_x+i \hat{\sigma}_y}{2}+\text { h.c. }\right)
\end{equation}
考虑周期边界条件,体哈密顿量可以写为:
\begin{equation}
	\hat{H}_{\text {bulk }}=\sum_{m=1}^N(v|m, B\rangle\langle m, A|+w|(\bmod N)+1, A\rangle\langle m, B|)+\text { h.c.. }
\end{equation} 
希望求解哈密顿量的本征值问题,本征方程为:
\begin{equation}
	\hat{H}_{\mathrm{bulk}}\left|\Psi_n(k)\right\rangle=E_n(k)\left|\Psi_n(k)\right\rangle
\end{equation}
其中$ n \in\{1, \ldots, 2 N\} $\footnote{这里是单粒子哈密顿量,不同格点之间的Hilbert空间为直积而非张量积关系,因此总维度是2N而不是$ 2^N $ }。为求解这个方程,对解做拟设,首先写出离散形式中对应的平面波:
\begin{equation}
	|k\rangle=\frac{1}{\sqrt{N}} \sum_{m=1}^N e^{i m k}|m\rangle, \quad \text { for } k \in\left\{\delta_k, 2 \delta_k, \ldots, N \delta_k\right\} \quad \text { with } \delta_k=\frac{2 \pi}{N}
\end{equation} 
k的取值为第一Brillouin区。由于哈密顿量具有$ internal\otimes sites $的形式,拟设解的形式为格点自由度与内禀自由度的张量积:$ \left|\Psi_n(k)\right\rangle=|k\rangle \otimes\left|u_n(k)\right\rangle ; \quad\left|u_n(k)\right\rangle=a_n(k)|A\rangle+b_n(k)|B\rangle $。其中$ \left|u_n(k)\right\rangle \in \mathscr{H}_{\text {internal }} $是动量空间中的体哈密顿量的本征态,定义为:
\begin{equation}
	\begin{aligned}
		\hat{H}(k) & =\left\langle k\left|\hat{H}_{\text {bulk }}\right| k\right\rangle=\sum_{\alpha, \beta \in\{A, B\}}\left\langle k, \alpha\left|H_{\text {bulk }}\right| k, \beta\right\rangle \cdot|\alpha\rangle\langle\beta| ; \\
		\hat{H}(k)\left|u_n(k)\right\rangle & =E_n(k)\left|u_n(k)\right\rangle .
		\end{aligned}
\end{equation} 
这个形式的解看起来和Bloch定理很像,但在此是离散情况,并且还有内禀自由度
\section{Peierls相变}
一维完美的格点是不稳定的,总会发生二聚化形成绝缘体,这称为Peierls相变。对于晶格间距为a的一维格点,紧束缚模型给出能带:
\begin{equation}
	\varepsilon_1(k)=-t\left[e^{i k a}+e^{-i k a}\right]=-2 t \cos (k a)
\end{equation}
假设发生二聚化,跃迁系数可以近似为:$ t^{\prime}(a \pm 2 b) \approx t \pm 2 b * \frac{\mathrm{d} t}{\mathrm{~d} a}=t \pm b * q_0=t \mp \Delta $,此时紧束缚模型给出两条能带:
\begin{equation}
	\begin{aligned}
		& H_2= {\left[\begin{array}{cc}
		0 & (t+\Delta)+(t-\Delta) \cdot e^{-i 2 k a} \\
		(t+\Delta)+(t-\Delta) \cdot e^{i 2 k a}
		\end{array}\right] } \\
		&= {\left[\begin{array}{cc}
		0 & 2 e^{-i k a}[t \cos (k a)+i \Delta \sin (k a)] \\
		2 e^{i k a}[t \cos (k a)-i \Delta \sin (k a)]
		\end{array}\right] } \\
		& \Rightarrow \quad \varepsilon_2(k)= \pm 2 \sqrt{t^2 \cos ^2(k a)+\Delta^2 \sin ^2(k a)}
		\end{aligned}
\end{equation} 
此时能带具有能隙。再考虑二聚化使得院子之间的能量发生的变换,一维声子的色散关系为:
\begin{equation}
	\omega=\sqrt{\frac{4 K}{M}}\left|\sin \frac{q a}{2}\right|
\end{equation}
二聚化相当于一个波矢为$ \frac{\pi}{a} $,振幅为b的格波,对应的能量为:
\begin{equation}
	\begin{aligned}
		\Delta E_{\ell} & =N \cdot \frac{1}{2} M \omega_{q=\frac{\pi}{a}}^2 b^2 \\
		& =N \cdot 2 K b^2
		\end{aligned}
\end{equation} 
由于能隙打开使得电子的能量改变量为:
\begin{equation}
	\begin{aligned}
		\Delta E_e & =\frac{N a}{2 \pi} \int_{-\pi / 2 a}^{\pi / 2 a}\left[\varepsilon_2(k)-\varepsilon_1(k)\right] \mathrm{d} k \\
		& =-2 t \frac{N a}{2 \pi} \int_{-\pi / 2 a}^{\pi / 2 a}\left[\sqrt{\cos ^2(k a)+\frac{\Delta^2}{t^2} \sin ^2(k a)}-\cos (k a)\right] \mathrm{d} k \\
		& =-2 t \frac{N}{\pi} \int_0^{\pi / 2}\left[\sqrt{\cos ^2(\theta)+\frac{\Delta^2}{t^2} \sin ^2(\theta)}-\cos (\theta)\right] \mathrm{d} \theta \\
		& =-2 t \frac{N}{\pi} \int_0^{\pi / 2}\left[\sqrt{\cos ^2(\theta)+\frac{\Delta^2}{t^2} \sin ^2(\theta)}-\cos (\theta)\right] \mathrm{d} \theta \\
		& =-2 t \frac{N}{\pi}\left[\int_0^{\pi / 2} \sqrt{1-(1-\lambda) \sin ^2(\theta)} \mathrm{d} \theta-1\right]
		\end{aligned}
\end{equation}
此积分杜对于$ \lambda $很小时可以近似给出:$ E(1-\lambda)=1+\lambda\left[a_1-\frac{1}{4} \ln (\lambda)\right]+\mathcal{O}\left(\lambda^2\right), \quad a_1=0.463 $,总能量的改变为声子与电子两部分的和:  
\begin{equation}
	\begin{aligned}
		&\begin{aligned}
		\Delta E_t & =\Delta E_{\ell}+\Delta E_e \\
		& =N \lambda[\alpha+\beta \ln \lambda]
		\end{aligned}\\
		&\alpha=\frac{2 K t^2}{q_0^2}-\frac{a_1 t}{\pi}, \quad \beta=\frac{t}{\pi}
		\end{aligned}
\end{equation}
稳定的构型对应的b由能量极值给出:$ \frac{\mathrm{d} \Delta E_t}{\mathrm{~d} \lambda}=0 $,则有:
\begin{equation}
	\begin{aligned}
		& \ln \lambda=-\frac{\alpha+\beta}{\beta} \\
		& \Delta E_t=N \beta \exp \left(-\frac{\alpha+\beta}{\beta}\right)
		\end{aligned}
\end{equation} 
可以看出只要$ \beta>0 $,对任意的$ \alpha,\beta $二聚化总是使得能量下降。  
\section{边界态}
当考虑热力学极限时,本征态出现两种行为:局域与非局域,分别称为边界态与体态。当取完全二聚化的极限时,两种态的区别是非常明显的,我们将这个极限为做过渡。\\

v=1, w=0的情况称为平凡的,此时本征态局域在二聚体内:
\begin{equation}
	v=1, w=0: \quad \hat{H}(|m, A\rangle \pm|m, B\rangle)= \pm(|m, A\rangle \pm|m, B\rangle)
\end{equation}
v=0, w=1的情况称为拓扑的,此时本征态局域在两个二聚体间:
\begin{equation}
	v=0, w=1: \quad \hat{H}(|m, B\rangle \pm|m+1, A\rangle)= \pm(|m, B\rangle \pm|m+1, A\rangle)
\end{equation}
很显然,此时哈密顿量不涉及边界上的两个态,因此出现两个零能态:
\begin{equation}
	v=0, w=1: \quad \hat{H}|1, A\rangle=\hat{H}|N, B\rangle=0
\end{equation}
\section{手征对称性}
一个哈密顿量是手征对称的,如果它满足下列条件:
\begin{equation}
	\hat{\Gamma} \hat{H} \hat{\Gamma}^{\dagger}=-\hat{H}
\end{equation}
其中$ \hat{\Gamma}  $是一个满足一些要求的幺正算符:
\begin{enumerate}
	\item 幺正且厄米:$ \hat{\Gamma}^{\dagger} \hat{\Gamma}=\hat{\Gamma}^2=1 $ 
	\item $ \hat{\Gamma} $在子格点上局域,即在不同元胞间没有矩阵元。
	\item Robus。固体物理中的哈密顿量总是依赖于一系列参数$ \xi \in \Xi $,它们对应的哈密顿量总是具有如下对称性:
\end{enumerate} 
\begin{equation}
	\forall \underline{\xi} \in \Xi: \quad \hat{\Gamma} \hat{H}(\underline{\xi}) \hat{\Gamma}=-\hat{H}
\end{equation}
可以定义两个正交投影算符:
\begin{equation}
	\hat{P}_A=\frac{1}{2}(\mathbb{I}+\hat{\Gamma}) ; \quad \hat{P}_B=\frac{1}{2}(\mathbb{I}-\hat{\Gamma}),
\end{equation}
哈密顿量可写为:
\begin{equation}
	\hat{P_A} \hat{H} \hat{P}_A=P_B \hat{H} \hat{P}_B=0 ; \quad \hat{H}=\hat{P_A} \hat{H} \hat{P}_B+\hat{P_B} \hat{H} \hat{P}_A
\end{equation}
具有手征对称性的哈密顿量,每个能量为E的本征态,总有一个手征伴:
\begin{equation}
	\hat{H}\left|\psi_n\right\rangle=E_n\left|\psi_n\right\rangle \quad \Longrightarrow \hat{H} \hat{\Gamma}\left|\psi_n\right\rangle=-\hat{\Gamma} \hat{H}\left|\psi_n\right\rangle=-\hat{\Gamma} E_n\left|\psi_n\right\rangle=-E_n \hat{\Gamma}\left|\psi_n\right\rangle
\end{equation}
这暗示了零能态的特殊。对于非零能态,由于不同本征值的本征态之间的正交性,显然有:
\begin{equation}
	\text { If } E_n \neq 0: \quad 0=\left\langle\psi_n|\hat{\Gamma}| \psi_n\right\rangle=\left\langle\psi_n\left|P_A\right| \psi_n\right\rangle-\left\langle\psi_n\left|P_B\right| \psi_n\right\rangle \text {. }
\end{equation}
零能态总是能选取为只在一个子格点上波函数非零,因为
\begin{equation}
	\text { If } \hat{H}\left|\psi_n\right\rangle=0: \quad \hat{H} \hat{P}_{A / B}\left|\psi_n\right\rangle=\hat{H}\left(\left|\psi_n\right\rangle \pm \hat{\Gamma}\left|\psi_n\right\rangle\right)=0
\end{equation}
SSH模型的手征对称性是显然的,选取:
\begin{equation}
	\hat{P}_A=\sum_{m=1}^N|m, A\rangle\left\langle n, A\left|; \quad \quad \hat{P}_B=\sum_{m=1}^N\right| m, B\right\rangle\langle n, B| .
\end{equation}
那么有$ \hat{\Sigma}_z=\hat{P}_A-\hat{P}_B $,可以验证
\begin{equation}
	\hat{P}_A \hat{H} \hat{P}_A=\hat{P}_B \hat{H} \hat{P}_B=0 ; \quad \Longrightarrow \quad \hat{\Sigma}_z \hat{H} \hat{\Sigma}_z=-\hat{H}
\end{equation} 
SSH模型的winding number可以定义为下述积分:
\begin{equation}
	\nu=\frac{1}{2 \pi} \int\left(\tilde{\mathbf{d}}(k) \times \frac{d}{d k} \tilde{\mathbf{d}}(k)\right)_z d k
\end{equation}
\section{SSH模型的边界态数}
定义哈密顿量的绝热演化,要求满足下列条件:
\begin{enumerate}
	\item 参数连续变化
	\item 重要的对称性保持不变
	\item 体能隙不关闭
\end{enumerate}
两个哈密顿量是绝热等价的若它们可以通过一个绝热演化关联起来。SSH模型中有两个拓扑不变量:winding number,$ N_a-N_b $,两者通过体边对应关系起来。
\section{边界态的严格求解}
考虑如下哈密顿量:
\begin{equation}
	\hat{H}=\sum_{m=1}^N\left(v_m|m, B\rangle\langle m, A|+\text { h.c. }\right)+\sum_{m=1}^{N-1}\left(w_m|m+1, A\rangle\langle m, B|+\text { h.c. }\right)
\end{equation} 
假设它存在零能态,现在求解零能波函数,本征方程为:
\begin{equation}
	\hat{H} \sum_{m=1}^N\left(a_m|m, A\rangle+b_m|m, B\rangle\right)=0 .
\end{equation}
这给出体和边的两组共2N个方程:
\begin{equation}
	m=1, \ldots, N-1: \quad v_m a_m+w_m a_{m+1}=0 ; \quad w_m b_m+v_{m+1} b_{m+1}=0
\end{equation}
\begin{equation}
	v_N a_N=0 ; \quad v_1 b_1=0
\end{equation}
第一组方程可以简单求解为:
\begin{equation}
	\begin{array}{ll}
		m=2, \ldots, N: & a_m=\prod_{j=1}^{m-1} \frac{-v_j}{w_j} a_1 ; \\
		m=1, \ldots, N-1: & b_m=\frac{-v_N}{w_m} \prod_{j=m+1}^{N-1} \frac{-v_j}{w_j} b_N .
		\end{array}
\end{equation}
而第二组方程给出:
\begin{equation}
	b_1=a_N=0
\end{equation}
这给出零能本征方程给出的本征波函数为零,因此不存在严格的零能态。但在热力学极限下,取$ N\to\infty $,且平均上来看$ v>w $ 有近似的零能态解,此时定义体平均:
\begin{equation}
	\overline{\log |v|}=\frac{1}{N-1} \sum_{m=1}^{N-1} \log \left|v_m\right| ; \quad \overline{\log |w|}=\frac{1}{N-1} \sum_{m=1}^{N-1} \log \left|w_m\right|
\end{equation} 
此时递推关系给出:
\begin{equation}
	\left|a_N\right|=\left|a_1\right| e^{-(N-1) / \xi} ; \quad\left|b_1\right|=\left|b_N\right| e^{-(N-1) / \xi} \frac{\left|v_N\right|}{\left|v_1\right|}
\end{equation}
\chapter{极化与Berry相}
此章以Rice-Mele模型为例子,从现代极化理论的角度讲解能带绝缘体的极化。Rice-Mele模型是SSH模型再加上交错的onsite能量,哈密顿量为
\begin{equation}
	\begin{aligned}
		\hat{H}=v \sum_{m=1}^N(|m, B\rangle\langle m, A|+\text { h.c. })+ & w \sum_{m=1}^{N-1}(|m+1, A\rangle\langle m, B|+\text { h.c. }) \\
		& +u \sum_{m=1}^N(|m, A\rangle\langle m, A|-| m, B\rangle\langle m, B|),
		\end{aligned}
\end{equation}
\section{Rice-Mele模型总的Wannier态}
能带哈密顿量的体本征态总是非定域的,在连续情况这可以从Bloch定理理解,Bloch态
\begin{equation}
	|\Psi(k)\rangle=|k\rangle \otimes|u(k)\rangle
\end{equation}
张成了占据子空间
\begin{equation}
	\hat{P}=\sum_{k \in B Z}|\Psi(k)\rangle\langle\Psi(k)|
\end{equation}
它具有规范自由度
\begin{equation}
	|u(k)\rangle \rightarrow e^{i \alpha(k)}|u(k)\rangle ; \quad \quad|\Psi(k)\rangle \rightarrow e^{i \alpha(k)}|\Psi(k)\rangle
\end{equation}
定义Wannier函数$ |w(j)\rangle \in \mathscr{H}_{\text {external }} \otimes \mathscr{H}_{\text {internal }} $,它具有正交性,也张成了占据子空间,晶格的平移对称性,且局域。可以证明Wannier态是Bloch态的傅立叶变换:
\begin{equation}
	|\Psi(k)\rangle=e^{-i \alpha(k)} \frac{1}{\sqrt{N}} \sum_{j=1}^N e^{i k j}|w(j)\rangle
\end{equation} 
等式右方是平移算符本征值为$ e^{-i k} $的本征态,和$ k^{\prime} \neq k $的Bloch态正交,同时又处于占据本征子空间,因此只能是$ |\Psi(k)\rangle $。Wannier函数由逆变换给出
\begin{equation}
	|w(j)\rangle=\frac{1}{\sqrt{N}} \sum_{k=\delta_k}^{N \delta_k} e^{-i j k} e^{i \alpha(k)}|\Psi(k)\rangle
\end{equation}  
Wannier函数的中心就是Berry相,
\begin{equation}
	\begin{aligned}
		& \hat{x}|w(0)\rangle=\frac{1}{2 \pi} \int_{-\pi}^\pi d k \sum_m m e^{i k m}|m\rangle \otimes|u(k)\rangle \\
		&=-\frac{i}{2 \pi}\left[\sum_m e^{i k m}|m\rangle \otimes|u(k)\rangle\right]_{-\pi}^\pi+\frac{i}{2 \pi} \int_{-\pi}^\pi d k \sum_m e^{i k m}|m\rangle \otimes\left|\partial_k u(k)\right\rangle \\
		&=\frac{i}{2 \pi} \int_{-\pi}^\pi d k \sum_m e^{i k m}|m\rangle \otimes\left|\partial_k u(k)\right\rangle .
		\end{aligned}
\end{equation}
Wannier心在模去倒格子的意义上规范不变。
\begin{equation}
	\langle w(j)|\hat{x}| w(j)\rangle=\frac{i}{2 \pi} \int_{-\pi}^\pi d k\left\langle u(k) \mid \partial_k u(k)\right\rangle+j
\end{equation}
其中对m的求和给出$ \delta_{k,k^\prime} $,极化率按照定义可以看出由第一项即Berry相给出,因此极化率的定义总是模去一个整数 
\section{反演对称与极化}
对于格点系统的宇称算符可以定义为
\begin{equation}
	\begin{aligned}
		\hat{\Pi}|k\rangle \otimes|u\rangle & =|-k\rangle \otimes \hat{\pi}|u\rangle \\
		\hat{\pi}^2 & =\hat{\pi}^{\dagger} \hat{\pi}=\mathbb{I}_{\text {internal }}
		\end{aligned}
\end{equation}
它对体动量空间的哈密顿量作用为
\begin{equation}
	\left\langle k\left|\hat{\Pi}\hat{H}_{\text {bulk }}\hat{\Pi}^{-1}\right| k\right\rangle =\hat{\pi}\left\langle-k\left| \hat{H}_{\text {bulk }} \right|-k\right\rangle\hat{\pi}^\dagger=\hat{\pi} \hat{H}(-k) \hat{\pi}^{\dagger}
\end{equation}
具有宇称对称性即要求存在一个在内禀Hilbert空间内的一个幺正、厄米算符,满足
\begin{equation}
	\hat{\pi} \hat{H}(-k) \hat{\pi}=\hat{H}(k)
\end{equation}
只考虑一条能带的情况
\begin{equation}
	\begin{aligned}
		\hat{H}(k)|u(k)\rangle=E(k)|u(k)\rangle \quad \Longrightarrow \hat{H}(-k) \hat{\pi}|u(k)\rangle=E(k) \hat{\pi}|u(k)\rangle ; \\
		\Longrightarrow|u(-k)\rangle=e^{i \phi(k)} \hat{\pi}|u(k)\rangle .
		\end{aligned}
\end{equation}
对于$k=0,k=\pi $的空间反演对称点,我们说它具有确定的宇称
\begin{equation}
	\begin{aligned}
		|u(0)\rangle & =p_0|u(0)\rangle ; & |u(\pi)\rangle & =p_\pi|u(\pi)\rangle, \\
		\text { with } \quad p_0 & = \pm 1 ; & p_\pi & = \pm 1 .
		\end{aligned}
\end{equation}
\chapter{Adiabatic charge pumping, Rice-Mele Model}
考虑SSH模型中增加onsite能量,并且哈密顿量中的参数依赖于同一参数t
\begin{equation}
	\begin{array}{r}
		\hat{H}(t)=v(t) \sum_{m=1}^N(|m, B\rangle\langle m, A|+\text { h.c. })+w(t) \sum_{m=1}^{N-1}(|m+1, A\rangle\langle m, B|+\text { h.c. }) \\
		+u(t) \sum_{m=1}^N(|m, A\rangle\langle m, A|-| m, B\rangle\langle m, B|)
		\end{array}
\end{equation}

	\bibliographystyle{IEEEtran}
	\bibliography{Ref}
\end{document}