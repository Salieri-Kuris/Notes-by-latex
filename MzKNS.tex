
\usepackage[UTF8]{ctex,hyperref}
\begin{document}

\begin{Large}
    \textsf{\textbf{2d$ M_z $  k-NS的不变量}}

\end{Large}

对于2d$ M_z $  k space non-symmorphic $ \overset{\sim}{M_z}:k_x,k_y\to k_x+\pi,k_y $,它对于Wilson Loop Spectrum的限制是$ \left\{\nu_y^i\left(k_x\right)\right\} \stackrel{\overset{\sim}{M_z}}{=}\left\{-\nu_y^i\left(k_x\right)\right\} \bmod 1 $ \\。它和陈数是兼容的,但这个$ \overset{\sim}{M_z} $需要$ kx,kx+\pi $的简并,奇数陈数的边界态无法满足这个条件,因此分类是$ 2\mathbb{Z} $。\\
当加入Class D的PHS后,$ \overset{\sim}{M_z}C:(k_x,k_y)\to(-k_x+\pi,-k_y) $在$ k_x=\pm\frac{\pi}{2} $线上可以视为一个Effective的Class D PHS,使得这两条线上的Berry Phase$ \nu(\pm\frac{\pi}{2}) $  quantized to $ 0,\pi $,同时PHS限制$ \nu(\frac{\pi}{2}) = \nu(-\frac{\pi}{2}) mod 2\pi$。系统的陈数满足$ chern =  \frac{2*(\nu(\frac{\pi}{2})-\nu(-\frac{\pi}{2}))}{2\pi}$。当陈数为零时,$ \nu(\pi) = \nu(-\pi) = \nu(\frac{\pi}{2}) = \nu(-\frac{\pi}{2}) = 0 \text{ or } \pi$时,为$ \mathbb{Z}_2 $,这个$ \mathbb{Z}_2 $同时也是Class D的weak invariant。所以在这里我不确定这个$ \mathbb{Z}_2 $是Strong还是Weak \\          
任何一个k space non-symmorphic的对称性总可以等同为在扩大的元胞中满足下列要求的一组对称性
\begin{enumerate}
    \item 一个half translation $ T_{\frac{1}{2}} $ 
    \item 一个空间对称性或者onsite的对称性$ U $
    \item U和$ T_{\frac{1}{2}} $反对易
\end{enumerate}
$ T_{\frac{1}{2}} $是一个$ 2\mathbb{Z} $ real space nonsymmorphic对称性,它在Class D中的分类\href{https://journals.aps.org/prb/pdf/10.1103/PhysRevB.93.195413#page=19.70}{在这里给出},是$ \mathbb{Z}\oplus\mathbb{Z}_2 $,在此处$ U $是Onsite的,它导致能谱处处两重简并,$ \mathbb{Z}\to2\mathbb{Z} $,而文章中的equation(35)给出的$ \mathbb{Z}_2 $不变量有关系$ \nu_+ = \nu_- $,最终还是一个$ \mathbb{Z}_2 $,最终分类为$ 2\mathbb{Z}\oplus\mathbb{Z}_2 $     
\end{document}