\documentclass[12pt, a4paper, oneside]{ctexbook}
\usepackage{amsmath, amsthm, amssymb, bm, graphicx, hyperref, mathrsfs}
\usepackage{ctex}
\bibliographystyle{IEEEtran}
\bibliography{Ref}
\title{{\Huge{\textbf{那些我曾经懂过的问题}}}\\}
\author{Salieri}
\date{\today}
\linespread{1.5}
\newtheorem{theorem}{定理}[section]
\newtheorem{definition}[theorem]{定义}
\newtheorem{lemma}[theorem]{引理}
\newtheorem{corollary}[theorem]{推论}
\newtheorem{example}[theorem]{例}
\newtheorem{proposition}[theorem]{命题}
\def\mt{\mathscr{T}}
\def\ms{\mathscr{S}}
\def\mf{\mathscr{F}}
\def\ml{\mathscr{L}}
\def\R{\mathbb{R}}
\def\de{definition}
\def\th{theorem}

\newcommand{\fp}[2]{\frac{\partial#1}{\partial#2}}
\begin{document}
	\maketitle
	
	\pagenumbering{roman}
	\setcounter{page}{1}
	
	\begin{center}
		\Huge\textbf{前言}
	\end{center}~\
	
	此笔记总结我在学习过程中遇到的各种细节问题。
	~\\
	\begin{flushright}
		\begin{tabular}{c}
			Salieri\\
			\today
		\end{tabular}
	\end{flushright}
	
	\newpage
	\pagenumbering{Roman}
	\setcounter{page}{1}
	\tableofcontents
	\newpage
	\setcounter{page}{1}
	\pagenumbering{arabic}
	\chapter{凝聚态场论、量子多体}
    \section{电、声子有效相互作用}
    \chapter{量子信息}
    \section{Multipartite Entanglement}
	\chapter{量子场论}
    \section{同一粒子的不同表示}
	\section{Dirac,Weyl and majorana fermions}\cite{pal2011dirac}
	
    \section{Pology}
	\section{Noether定理与local对称性}
    \section{规范场的量子化:QED,非阿贝尔,与凝聚态中的电磁场路径积分}
	\chapter{张量网络}
	
\end{document}
