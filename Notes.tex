\documentclass[12pt, a4paper, oneside]{ctexbook}
\usepackage{amsmath, amsthm, amssymb, bm, graphicx, hyperref, mathrsfs}
\usepackage{ctex}

\title{{\Huge{\textbf{那些我曾经懂过的问题}}}\\}
\author{Salieri}
\date{\today}
\linespread{1.5}
\newtheorem{theorem}{定理}[section]
\newtheorem{definition}[theorem]{定义}
\newtheorem{lemma}[theorem]{引理}
\newtheorem{corollary}[theorem]{推论}
\newtheorem{example}[theorem]{例}
\newtheorem{proposition}[theorem]{命题}
\def\mt{\mathscr{T}}
\def\ms{\mathscr{S}}
\def\mf{\mathscr{F}}
\def\ml{\mathscr{L}}
\def\R{\mathbb{R}}
\def\de{definition}
\def\th{theorem}
\def\gm{$ \gamma $矩阵 }
\newcommand{\fp}[2]{\frac{\partial#1}{\partial#2}}
\begin{document}
	\maketitle
	
	\pagenumbering{roman}
	\setcounter{page}{1}
	
	\begin{center}
		\Huge\textbf{前言}
	\end{center}~\
	
	此笔记总结我在学习过程中遇到的各种细节问题。
	~\\
	\begin{flushright}
		\begin{tabular}{c}
			Salieri\\
			\today
		\end{tabular}
	\end{flushright}
	
	\newpage
	\pagenumbering{Roman}
	\setcounter{page}{1}
	\tableofcontents
	\newpage
	\setcounter{page}{1}
	\pagenumbering{arabic}
	\chapter{凝聚态场论、量子多体}
    \section{电、声子有效相互作用}
    \chapter{量子信息}
    \section{Multipartite Entanglement}
	\chapter{量子场论}
    \section{同一粒子的不同表示}
	\section{Dirac,Weyl and Majorana fermions}\cite{pal2011dirac}
	简单来讲,Dirac fermion是Dirac方程的一般解,Majorana fermion是Dirac方程的“实”解,Weyl fermion是无质量Dirac方程的解。\\
	\subsection{Dirac方程及其解}
	Dirac方程定义为:
	\begin{equation}
		\left(i \gamma^\mu \partial_\mu-m\right) \Psi=0\label{de}
	\end{equation}
	它可以看作具有如下哈密顿量的薛定谔方程:
	\begin{equation}
		H=\gamma^0\left(\gamma^i p^i+m\right)
	\end{equation}
	$\gamma$ 矩阵定义为:
	\begin{equation}
		\begin{aligned}
			& {\left[\gamma^\mu, \gamma^\nu\right]_{+}=2 g^{\mu \nu},} \\
			& \gamma_0 \gamma_\mu \gamma_0=\gamma_\mu^{\dagger}
			\end{aligned}
	\end{equation}
	可以看到,当$ \gamma $矩阵为纯虚时,Dirac方程为实方程,有实解。我们可以找到一组\gm 满足这样的条件,这个表象称为Majorana表象。
	\begin{equation}
		\begin{array}{ll}
			\widetilde{\gamma}^0=\left[\begin{array}{cc}
			0 & \sigma^2 \\
			\sigma^2 & 0
			\end{array}\right], \quad \widetilde{\gamma}^1=\left[\begin{array}{cc}
			i \sigma^1 & 0 \\
			0 & i \sigma^1
			\end{array}\right], \\
			\tilde{\gamma}^2=\left[\begin{array}{cc}
			0 & \sigma^2 \\
			-\sigma^2 & 0
			\end{array}\right], \quad \tilde{\gamma}^3=\left[\begin{array}{cc}
			i \sigma^3 & 0 \\
			0 & i \sigma^3
			\end{array}\right],
			\end{array}
	\end{equation}
	其中$ \sigma^i $为Pauli矩阵。在此表象下写出Dirac方程,我们有实解:
	\begin{equation}
		\widetilde{\psi}=\tilde{\psi}^{\star}\label{rc}
	\end{equation} 
	此解即代表Majorana fermion。\gm 不同表示之间通过幺正变换相联系:
	\begin{equation}
		\gamma^\mu=U \tilde{\gamma}^\mu U^{\dagger}
	\end{equation}
	此时解$ \tilde{\Psi} $与Majorana表象下的解也通过一个幺正矩阵相联系:
	\begin{equation}
		\Psi=U \tilde{\Psi}
	\end{equation} 
	实解条件\eqref{rc}此时为
	\begin{equation}
		\psi=U U^{\top} \psi^{\star}
	\end{equation}
	我们一般不直接使用幺正矩阵$ U $,而是转而定义如下矩阵:
	\begin{equation}
		U U^{\top}=\gamma_0 C
	\end{equation} 
	由此定义协变共轭(协变性将在稍后证明):
	\begin{equation}
		\hat{\Psi} \equiv \gamma_0 C \Psi^{\star}
	\end{equation}
	此时实解条件\eqref{rc}可以写为:
	\begin{equation}
		\hat{\psi}=\psi\label{grc}
	\end{equation}
	\subsection{Fourier展开}
	一个Majorana fermion解在一般的表象下的Fourier展开为:
	\begin{equation}
		\psi(x)=\sum_s \int_p\left(a_s(p) u_s(p) e^{-i p \cdot x}+a_s^{\dagger}(p) v_s(p) e^{+i p \cdot x}\right)
	\end{equation}
	v和u互为协变共轭$ v_s(p)=\gamma_0 C u_s^{\star}(p) $
	\subsection{矩阵C的一些性质}
	除了通过找到实解对应的Majorana表象外,我们还可以通过研究 C矩阵本身的性质来定义一组C矩阵,再由此找到不依赖于表象的“实解”条件。\\
	简单观察可以发现,C矩阵满足如下性质:
	\begin{equation}
		C^{-1} \gamma_\mu C=-\left(U \tilde{\gamma}_\mu U^{\dagger}\right)^{\top}=-\gamma_\mu^{\top}
	\end{equation} 
	事实上,对于任何表象的\gm 我们总可以找到满足上式的C矩阵,此式可以直接称为C矩阵的定义,由此我们有一般情况的实解条件\eqref{grc}
    \section{Pology}
	\section{Noether定理与local对称性}
    \section{规范场的量子化:QED,非阿贝尔,与凝聚态中的电磁场路径积分}
	\chapter{张量网络}
	\bibliographystyle{IEEEtran}
	\bibliography{Ref}
\end{document}
