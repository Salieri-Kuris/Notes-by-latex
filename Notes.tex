\documentclass[12pt, a4paper, oneside]{ctexbook}
\usepackage{amsmath, amsthm, amssymb, bm, graphicx, hyperref, mathrsfs}
\usepackage{ctex}

\title{{\Huge{\textbf{那些我曾经懂过的问题}}}\\}
\author{Salieri}
\date{\today}
\linespread{1.5}
\newtheorem{theorem}{定理}[section]
\newtheorem{definition}[theorem]{定义}
\newtheorem{lemma}[theorem]{引理}
\newtheorem{corollary}[theorem]{推论}
\newtheorem{example}[theorem]{例}
\newtheorem{proposition}[theorem]{命题}
\setcounter{tocdepth}{1}%目录显示到第一级,section
\setcounter{secnumdepth}{2}%编号到第二级,subsection
\def\mt{\mathscr{T}}
\def\ms{\mathscr{S}}
\def\mf{\mathscr{F}}
\def\ml{\mathscr{L}}
\def\R{\mathbb{R}}
\def\de{definition}
\def\th{theorem}
\def\gm{$ \gamma $矩阵 }
\newcommand{\fp}[2]{\frac{\partial#1}{\partial#2}}
\begin{document}
	\maketitle
	
	\pagenumbering{roman}
	\setcounter{page}{1}
	
	\begin{center}
		\Huge\textbf{前言}
	\end{center}~\
	
	此笔记总结我在学习过程中遇到的各种细节问题。
	~\\
	\begin{flushright}
		\begin{tabular}{c}
			Salieri\\
			\today
		\end{tabular}
	\end{flushright}
	
	\newpage
	\pagenumbering{Roman}
	\setcounter{page}{1}
	\tableofcontents
	\newpage
	\setcounter{page}{1}
	\pagenumbering{arabic}
	\chapter{凝聚态场论、量子多体}
    \section{电、声子有效相互作用}
    \chapter{量子信息}
    \section{Multipartite Entanglement}
	\chapter{量子场论}
    \section{同一粒子的不同表示}
	\section{Dirac,Weyl and Majorana fermions}\cite{pal2011dirac}
	简单来讲,Dirac fermion是Dirac方程的一般解,Majorana fermion是Dirac方程的“实”解,Weyl fermion是无质量Dirac方程的解。\\
	\subsection{Dirac方程及其解}
	Dirac方程定义为:
	\begin{equation}
		\left(i \gamma^\mu \partial_\mu-m\right) \Psi=0\label{de}
	\end{equation}
	它可以看作具有如下哈密顿量的薛定谔方程:
	\begin{equation}
		H=\gamma^0\left(\gamma^i p^i+m\right)
	\end{equation}
	$\gamma$ 矩阵定义为:
	\begin{equation}
		\begin{aligned}
			& {\left[\gamma^\mu, \gamma^\nu\right]_{+}=2 g^{\mu \nu},} \\
			& \gamma_0 \gamma_\mu \gamma_0=\gamma_\mu^{\dagger}
			\end{aligned}
	\end{equation}
	可以看到,当$ \gamma $矩阵为纯虚时,Dirac方程为实方程,有实解。我们可以找到一组\gm 满足这样的条件,这个表象称为Majorana表象。
	\begin{equation}
		\begin{array}{ll}
			\widetilde{\gamma}^0=\left[\begin{array}{cc}
			0 & \sigma^2 \\
			\sigma^2 & 0
			\end{array}\right], \quad \widetilde{\gamma}^1=\left[\begin{array}{cc}
			i \sigma^1 & 0 \\
			0 & i \sigma^1
			\end{array}\right], \\
			\tilde{\gamma}^2=\left[\begin{array}{cc}
			0 & \sigma^2 \\
			-\sigma^2 & 0
			\end{array}\right], \quad \tilde{\gamma}^3=\left[\begin{array}{cc}
			i \sigma^3 & 0 \\
			0 & i \sigma^3
			\end{array}\right],
			\end{array}
	\end{equation}
	其中$ \sigma^i $为Pauli矩阵。在此表象下写出Dirac方程,我们有实解:
	\begin{equation}
		\widetilde{\psi}=\tilde{\psi}^{\star}\label{rc}
	\end{equation} 
	此解即代表Majorana fermion。\gm 不同表示之间通过幺正变换相联系:
	\begin{equation}
		\gamma^\mu=U \tilde{\gamma}^\mu U^{\dagger}
	\end{equation}
	此时解$ \tilde{\Psi} $与Majorana表象下的解也通过一个幺正矩阵相联系:
	\begin{equation}
		\Psi=U \tilde{\Psi}
	\end{equation} 
	实解条件\eqref{rc}此时为
	\begin{equation}
		\psi=U U^{\top} \psi^{\star}
	\end{equation}
	我们一般不直接使用幺正矩阵$ U $,而是转而定义如下矩阵:
	\begin{equation}
		U U^{\top}=\gamma_0 C
	\end{equation} 
	由此定义协变共轭(协变性将在稍后证明):
	\begin{equation}
		\hat{\Psi} \equiv \gamma_0 C \Psi^{\star}\label{ccg}
	\end{equation}
	此时实解条件\eqref{rc}可以写为:
	\begin{equation}
		\hat{\psi}=\psi\label{grc}
	\end{equation}
	\subsection{Fourier展开}
	一个Majorana fermion解在一般的表象下的Fourier展开为:
	\begin{equation}
		\psi(x)=\sum_s \int_p\left(a_s(p) u_s(p) e^{-i p \cdot x}+a_s^{\dagger}(p) v_s(p) e^{+i p \cdot x}\right)
	\end{equation}
	v和u互为协变共轭$ v_s(p)=\gamma_0 C u_s^{\star}(p) $
	\subsection{矩阵C的一些性质}
	除了通过找到实解对应的Majorana表象外,我们还可以通过研究 C矩阵本身的性质来定义一组C矩阵,再由此找到不依赖于表象的“实解”条件。\\
	简单观察可以发现,C矩阵满足如下性质:
	\begin{equation}
		C^{-1} \gamma_\mu C=-\left(U \tilde{\gamma}_\mu U^{\dagger}\right)^{\top}=-\gamma_\mu^{\top}
	\end{equation} 
	事实上,对于任何表象的\gm 我们总可以找到满足上式的C矩阵,此式可以直接称为C矩阵的定义,由此我们有一般情况的实解条件\eqref{grc}
    同时,很容易验证C矩阵在任意表象下都是完全反对称矩阵。
	\subsection{实条件的洛伦兹不变性}
	此节将阐述为什么称\eqref{ccg}为协变共轭,我们取Lorentz变换的生成元为$ \sigma^{\mu\nu}=\frac{i}{2}[\gamma_\mu,\gamma_\nu]  $,注意,这并非Pauli矩阵。一个fermion场的变换为:
	\begin{equation}
		\Psi^{\prime}\left(x^{\prime}\right)=\exp \left(-\frac{i}{4} \omega^{\mu \nu} \sigma_{\mu \nu}\right) \Psi(x)
	\end{equation}
	对上式取复共轭并左乘$ \gamma_0C $,注意到\gm 的性质,可以推出:
	\begin{equation}
		\widehat{\Psi}^{\prime}\left(x^{\prime}\right)=\exp \left(-\frac{i}{4} \omega^{\mu \nu} \sigma_{\mu \nu}\right) \widehat{\Psi}(x)
	\end{equation} 
	由此可以看出,协变共轭场于原场具有相同的洛伦兹变换规则,所以实解条件是洛伦兹不变的。
	\subsection{左与右}
	在处理费米场时,我们通常会遇到两个和左右有关的概念,螺旋度(helicity)和手性(chirality),两者通常并不相同,但在某些情况下又有关联,本节将阐述这种关联。
	\paragraph*{螺旋度}
	满足Dirac方程的费米子额度螺旋度定义为:
	\begin{equation}
		h_p=\frac{\boldsymbol{\Sigma} \cdot \boldsymbol{p}}{\mathrm{p}}
	\end{equation}
	$ \boldsymbol{\Sigma} $为自旋矩阵。很显然,$ h_p $的本征值为$ \pm $  1,分别对应“右手”和“左手”。可以验证  $ h_p $与哈密顿量对易,也就是说它是一个守恒量,并且其点乘结构也保证了旋转不变形,但可以验证对于有质量粒子,螺旋度在推动下是改变的。
	一个简单的论证是考虑螺旋度的物理意义:自旋在动量方向上的投影,考虑在粒子速度方向上一运动更快的参考系,此时动量反号,但自旋投影在该boost下不改变,因此螺旋度改变。
	上面的论述依赖于速度更快的参考系,这对无质量粒子是不可能的,因此这暗示无质量粒子的螺旋度是一个真正的洛伦兹不变量。
	\paragraph*{手性}
	可以额外定义一个矩阵$ \gamma_5 $,它满足:
	\begin{equation}
		\left[\gamma_5, \gamma_\mu\right]_{+}=0 \quad \forall \mu
	\end{equation} 
	一个显然的解是:
	\begin{equation}
		\gamma_5=i \gamma^0 \gamma^1 \gamma^2 \gamma^3
	\end{equation}
	给上式乘上额外的相因子也是满足定义的,上述的选择保证了$ \gamma_5 $有如下性质:
	\begin{equation}
		\gamma_5^{\dagger}=\gamma_5, \quad\left(\gamma_5\right)^2=1
	\end{equation} 
	后者保证了如下定义的两个矩阵确实是投影算符:
	\begin{equation}
		L=\frac{1}{2}\left(1-\gamma_5\right), \quad R=\frac{1}{2}\left(1+\gamma_5\right)
	\end{equation}
	两者的本征空间里的矢量分别称为“左手”的和“右手”的。每个Dirac旋量也总可以拆分成两者之和。可以验证,$ \gamma_5 $与洛伦兹变换的生成元对易,但与哈密顿量不对易,这是由于质量相含一个\gm ,因此和$ \gamma_5 $反对易。
	\subsection{Wyle fermion}
	如前文所述,质量项同时阻碍了螺旋度和手性成为性质良好的物理量,因此我们考虑无质量费米子。由于$ \gamma_5 $的存在以及Schur引理,
	一般的Dirac方程解并非Lorentz群的不可约表示,而手性解是可约的,称为Wyle fermion。一个一般的费米子是可约表示$ \frac{1}{2}\oplus\frac{1}{2} $,
	即一个左手Wyle一个右手Wyle。上述说法对螺旋度同样适用,因为无质量时两者相同。
	\subsection{从Wyle费米子到Majorana 与Dirac费米子}
	Majorana fermion是含有质量的,因此必须包含左右手两部分,考虑到实条件,这必须是一个Wyle场与它的协变共轭:
	\begin{equation}
		\psi(x)=\chi(x)+\widehat{\chi}(x)
	\end{equation}
	而Dirac fermion的区别就是没有实解条件,因此两个左手、右手场之间独立:
	\begin{equation}
		\Psi(x)=\chi_1(x)+\widehat{\chi}_2(x)
	\end{equation}
	\section{Pology}
	\section{Noether定理与local对称性}
    \section{规范场的量子化:QED,非阿贝尔,与凝聚态中的电磁场路径积分}
	\chapter{张量网络}
	\bibliographystyle{IEEEtran}
	\bibliography{Ref}
\end{document}
